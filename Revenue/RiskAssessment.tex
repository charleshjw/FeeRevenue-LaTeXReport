\subsection{Background}
The risk inventory is part of a broader risk identification and risk materiality
process intended to drive the identification of material risks for CCAR scenario
generation process for each business.\\

This enterprise wide risk inventory was developed leveraging the Bank's Business
As Usual risk management and Line Of Business processes. Additional risks were
also identified through review during working sessions with key business line
CROs and business partner leads.

\subsection{Objective}
The objective of the risk inventory is to aggregate the Bank's financial and
non-financial risks to financial performance in a manner that is aligned to the
Bank's aggregated risk taxonomy developed for the identification and materiality
of key risks for the firm. (Note - this preliminary risk inventory does not
represent the complete universe risks at this time).\\

This risk inventory will be used to identify the segmentation and granulaity we
want to model for each business. The end goal of this process is to perform a
materiality assessment for all identified risks, in order to identify those that
would result in a material financial and/or non-financial impact to the Bank
from a capital adequacy perspective.

\subsection{Questions for Business Team}
Assess severity and probability of bank's top 45 business risks and determine
their impact to fee revenue. Questions need to be answered are:
\begin{itemize}
  \item How should we segment revenue to better reflect risks?
  \item How should we model revenue to capture the distinctive nature of
  material business risk?
  \item What data are required?
  \item	Is there any other material risk not included in top 45 risks?
  \item Based on business intuition and historical experience, which risks would
  predominantly impact the business and to what extent under each of the
  Supervisory Scenarios from the Fed and the BHC Stress Scenario?
  \item How should we calculate overlay adjustments/ management judgments to
  account for these risks? Are there other business specific risks that are not
  included in this list?
\end{itemize}

To get oriented with the next couple of pages, we have described the risks that
are considered material from a BNY Mellon perspective. To the right, we have a
materialy assesment scorecard which is stated in terms of severity rating; 1
being least severe and 5 being most.

\begin{enumerate}
  \item Negative financial impact due to lack of economic growth and\/or
  deflationary pressures
  \item Customized commitments are not commercially viable and\/or operationally
  feasible
  \item Ineffective growth strategy
  \item Sactions resulting in frozen assets
  \item Business model disruption due to market entrants and technology
  \item Restitution risk
  \item Inadvertently factilitating illegal activity
  \item Poor performance by bank asset managers driving voluntary action to
  support client investments
  \item Internal fraud, including control breaches or inaccurate recording,
  manipulation and arbitrage of Bank systems
  \item Improper fund administration activities
  \item Resource contention impairs strategic initiatives or ability to meet
  contractual obligations
  \item Ponzi scheme in investment managmeent funds distributed by the company
  \item Inaccurate data, books and records which are not in line with corporate
  and\/or regulatory requirements
  \item Undue reliance on vendors for critical operations, or ineffective
  oversight/vendor management, vendor techonology disruptions interrupting
  service delivery or regulatory requirements
  \item Insufficient management information
  \item Unclear or misunderstood instructions or documents from clients leading
  to financial impact
  \item Manual data aggregation errors leading to inaccurate reporting
  \item One of the firms products or processes is utilized to perpetrate an
  external fraud
  \item Inability to follow procedure in executing client transactions
  \item Unsuitable processes, systems or lack of appropriate skillset to operate
  business model
  \item Failure to comply with or interpret contracts and inadequate due
  diligence at account opening
  \item Inadequate disaster recovery and business continuity planning, testing
  or controls
  \item Failed execution of transformation projects and\/or large scale
  strategic project initiatives
  \item Ineffective due diligence, project complexity, increased deal
  complexity, customization or insufficient resources
  \item Ineffective model validation and testing resulting in operational and
  financial impact
  \item Improper oversight or inadequate process over 3rd party investment
  managers
  \item Cyber attack
  \item Confidential data breach
  \item Widespread production/systems outages due to ineffective testing/system
  upgrades
  \item Damage to data centers and\/or other back up centers
  \item Unclear or inaccurate marketing materials
  \item Inadequate compliance program
  \item Unethical customer treatment
  \item Faulty algorithm resulting in runaway/unintended trades
  \item Incorrect tax treatment
  \item Unsuitable sale of securities\/product advice or mismanagement of client
  assets
  \item Liability for lost fund assets from sub-custodian actions
  \item Breaches of fund\/trust laws or provisions
  \item Large counterparty default
  \item Sovereign credit crisis
  \item Idiosyncratic credit default and\/or redemption of securities in
  investments
  \item Adverse changes in interest rates impact capital (OCI)
  \item Adverse changes in interest rates impact Economic Value of Equity
  \item Inadequate contingent funding
  \item Wrong way risk
\end{enumerate}